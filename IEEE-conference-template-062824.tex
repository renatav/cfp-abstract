\documentclass[conference]{IEEEtran}
\IEEEoverridecommandlockouts
% The preceding line is only needed to identify funding in the first footnote. If that is unneeded, please comment it out.
%Template version as of 6/27/2024

\usepackage{cite}
\usepackage{amsmath,amssymb,amsfonts}
\usepackage{algorithmic}
\usepackage{graphicx}
\usepackage{textcomp}
\usepackage{xcolor}
\usepackage{url}
\def\BibTeX{{\rm B\kern-.05em{\sc i\kern-.025em b}\kern-.08em
    T\kern-.1667em\lower.7ex\hbox{E}\kern-.125emX}}
\begin{document}

\newif\ifrev
%%%%%%%%%%%%%%%%%%
% COMMENT OUT NEXT LINE TO HIDE COMMENTS
\revtrue
%%%%%%%%%%%%%%%%%%
\ifrev
  \newcommand{\yanyan}[1]{{\color{blue} [Yanyan: #1]}}
  \newcommand{\red}[1]{\textcolor{red}{#1}}
  \newcommand{\todo}[1]{\textcolor{red}{\textbf{Anyone todo}: #1}}
\else
  \newcommand{\yanyan}[1]{}
  \newcommand{\red}[1]{}
  \newcommand{\todo}[1]{}
\fi

\title{Rooting Out Atoms Of Confusion -- A Call To Action}

\author{Justin Cappos${}^{\rm ny}$ \qquad Dan Gopstein${}^{\rm ny}$ \qquad Renata Vaderna${}^{\rm ny}$ \qquad Yanyan Zhuang${}^{\rm co}$\\
${}^{\rm ny}$\,New York University 
${}^{\rm co}$\,University of Colorado, Colorado Springs  %\thanks{This work is supported in part by \yanyan{Justin complete}.}
}

\maketitle

\begin{abstract}
The recent work by our team has empirically 
validated the existence of small patterns in C code
that can interfere with program comprehension.
These small code patterns, named atoms of confusion 
(or atoms for short), are correlated with many bugs in software. This finding has been replicated by many 
research teams. For example, atoms have been validated 
in different programming languages including Java and 
Python, and in a variety of ways such as using an 
electroencephalogram (EEG) device.

Since we have a growing community of academic teams, 
we have set out to put our work into practice by 
bridging the gap between academia and industry.  We 
are currently developing data 
sharing and visualization tools, which are working with the Linux Foundation 
to host in a neutral, industry facing manner. \yanyan{This sentence has grammar issues?}  We 
also have been working to deploy atom parsers in the Linux 
kernel. In summary, we plan to build a broader 
community through academia-industry collaboration,  
encourage, engage, and enhance community connections 
to grow into a large open-source ecosystem.  We invite you to join us!
\end{abstract}

\begin{IEEEkeywords}
Atoms of Confusion, Code Comprehension, Community Effort, Open Source
\end{IEEEkeywords}

\section{Introduction}

Software bugs represent a significant challenge worldwide, 
with a staggering cost of approximately \$2.08 trillion in the 
US alone in 2020, as reported by the Consortium for 
Information \& Software Quality (CISQ) \cite{CPSQ2020}. 
Addressing these bugs efficiently before they cause damage is 
crucial in both industry and academia.
%
Code review serves as a cornerstone of quality assurance in 
both open source and proprietary projects. Research highlights 
that inadequate code reviews are linked to diminished software 
quality, the emergence of anti-patterns, and increased 
post-release defects. %\yanyan{Should have a citation here?} 
Ensuring that software is understandable to 
humans is crucial yet challenging, often approached ad hoc.

Our research has delved into the causes of code confusion for 
human developers. We have identified and empirically validated 
small C code patterns, termed \emph{atoms of confusion}, which 
programmers often misinterpret~\cite{gopstein2017understanding}. Analyzing 14 large C projects 
revealed hundreds of thousands of these atoms, highly 
correlated with both comments and bug fixes, underscoring 
their profound impact on software reliability 
\cite{gopstein2018prevalence}.
%
This body of work has inspired a breadth of subsequent studies 
across various programming languages, employing methodologies 
like eye-tracking and EEG to probe deeper into how programmers 
interact with these confusing elements 
\cite{langhout2021atoms, mendes2021bohr, mendes2022dazed, 
torres2023investigation, dacosta2023seeing, 
Manor2018AtomsConfusionSwift, yeh2017detecting}.

As this research started to gain momentum, we initiated 
efforts to enhance community collaboration. By fostering 
discussions, sharing datasets, and initiating research into 
unified standardization of what constitutes an atom of 
confusion, we aim to streamline the process for researchers to 
conduct future atoms-related experiments. Additionally, we are 
committed to expanding awareness and fostering interest in 
this novel approach of examining the causes of code confusion 
and bugs, not just within academia, but also across the 
industry.

\section{Work by the Academic Community}

Ever since the work by Gopstein et al.~\cite{gopstein2017understanding}, over a dozen research teams, e.g., from the Netherlands and Brazil, have replicated experiments from our prior work in Java~\cite{langhout2021atoms, mendes2022dazed}, JavaScript~\cite{oliveira2019impact},  Swift~\cite{castor2018identifying}, Python~\cite{da2023seeing}, and the Android platform~\cite{tabosa2024dataset}. \yanyan{Add 1-2 more examples} A Senior Scientist from the University of Klagenfurt, Austria had reached out to our team. We collaborated and published a recent discovery that used the datasets from our work~\cite{gopstein2017understanding, zhuang2023developer} and designed a high-quality C programming ability assessment tool~\cite{glasauer2024c}.

\section{Enhancing Community Engagement}

With the community's enthusiasm, however, comes the different definitions and execution in these works that led to divergent uses of our original concept. In this section, we outline our effort to unify the research community and apply our work in collaboration with the industry.

\subsection{Community Meetings}
In our previous experience, organizing community meetings has 
proven to be effective in gathering individuals who 
are passionate about a research project. 
Since January 2024, we have initiated bimonthly meetings 
to cultivate a strong sense of community among researchers. 
%During these gatherings, we 
%encourage members to introduce themselves and to share both 
%past and recent research. 
%We ensure that lists of all 
%published papers are readily accessible, and distribute notes 
%and slides from presentations. 
Our goal is to establish 
a supportive platform where  participants can share 
their work and engage in constructive feedback exchanges 
within the community.
%
%Furthermore, we initiated discussions on challenges 
%particularly relevant to those studying code confusion, such 
%as conducting user studies. We hope that members of the 
%community who are new to this area can learn from and seek 
%guidance from experienced researchers who have conducted 
%several user studies.
%
Another challenge is the varied interpretations 
of what constitutes an atom. We organized 
meetings specifically focused on this issue, presenting 
multiple code snippets that have been seen by some as atoms of 
confusion, while others may not find them confusing. These 
discussions have sparked significant interest and motivated 
further exploration into understanding these divisive 
elements.


\subsection{A Framework for Classifying Atoms of Confusion}

We have identified a core issue arising from the absence of a 
formal definition of what constitutes an atom of confusion. 
While our previous classifiers \cite{gopstein2018prevalence} 
have shown correlations between atoms of confusion, bugs, and 
confusing code, they have not succeeded in providing clear 
definitions. Our goal is to develop a precise, easily 
extendable, and reproducible methodology for classifying 
specific code patterns as atoms. To this end, we are actively 
developing a framework that includes mathematical guidelines 
to determine what is and isn't an atom. This framework is not 
only designed to support hypothesis testing among community 
members but also to pave the way for future applications to 
explore atoms in various programming languages. This effort 
builds upon our findings that increased frequencies of 
comments and bug fixes often indicate the presence of atoms, 
reinforcing their significant impact on software reliability.

The main idea behind this ongoing research is to count 
occurrences of all AST subtrees within large open source 
projects, as well as in bugfix commits and commented code 
snippets, looking for patterns with high frequencies of 
bugfixes and comments. The next step involves applying a 
formal statistical method (link to method here) \yanyan{where is the link?} to clearly 
distinguish atom candidates from non-atoms. This approach aims 
to provide a more precise and objective answer to questions 
about questionable patterns, potentially uncovering additional 
confusing patterns that were previously unrecognized. 
Additionally, researchers should be able to apply different 
metrics, adding new dimensions or substituting the measures of 
bugfix and comment frequencies.


\subsection{Collaboration with the Industry}

We aim for our research on atoms of confusion to surpass 
academic borders and have a tangible impact in the industry. 
Our previous research has led to the detection and fix of a 
bug in the Linux kernel caused by one of the patterns we 
consider to be an atom of confusion. \yanyan{What bug is it? Be concrete.} This success demonstrates 
the potential for the practical impact of our work and 
highlights the importance of ongoing collaboration with 
industry.

\subsection{Coccinelle}

The tool widely used to detect issues and refactor certain 
parts of the Linux kernel code is Coccinelle. Coccinelle is a 
program matching and transformation engine that provides the 
Semantic Patch Language (SmPL) for specifying desired matches 
and transformations in C code. It was initially designed to 
document and automate collateral evolutions in device driver 
code. Unlike traditional patches, which are limited to 
specific instances, Coccinelle's semantic patches can modify 
hundreds of files across thousands of code sites by 
abstracting away specific details and variations \cite{coccinelle}.

Given its extensive use, and in direct collaboration with the 
maintainers of Coccinelle, we are implementing the detection 
of atoms using Coccinelle. While our primary focus is on 
deployment within the Linux kernel, Coccinelle can be used to 
process any C code. Consequently, the semantic patches we have 
developed to detect atoms can also be applied to other C 
codebases. We have implemented patches for the detection of 14 
out of 15 identified atoms, excluding only one that is beyond 
Coccinelle's capabilities. All these semantic patches are available in our GitHub repository \cite{githubcocci}. 

Additionally, to further encourage collaboration between 
academia and industry, we dedicated one of our community 
meetings to Coccinelle. We invited a key contributor to 
Coccinelle to present the tool to our community, with a 
specific focus on detecting atoms. This initiative brings us a 
step closer to bridging the gap between academic research and 
practical industry applications.

\subsection{OpenSSF and the Linux Foundation}

To further solidify our efforts and expand the impact of our 
research, we are planning to establish a collaboration with 
the Open Source Security Foundation (OpenSSF) and the Linux 
Foundation. These organizations provide a collaborative and 
structured environment that fosters innovation and 
standardization in open-source security. We have successfully 
collaborated with these organizations on other projects in the 
past, and with this established reputation, we aim to further 
promote the adoption of our methodologies and tools. Aligning 
with OpenSSF and the Linux Foundation will help ensure that 
our research on atoms of confusion can benefit a wider range 
of software projects across the industry.


{\scriptsize \bibliographystyle{abbrv} \bibliography{bibdata}}




\end{document}
